\documentclass[a4paper]{article}
\usepackage{header}
\usepackage{float}
\usepackage{cmap}

\newcommand\enumtocitem[3]{\item\textbf{#1}\addtocounter{#2}{1}\addcontentsline{toc}{#2}{\protect{\numberline{#3}} #1}}
\newcommand\defitem[1]{\enumtocitem{#1}{subsection}{\thesubsection}}
\newcommand\proofitem[1]{\enumtocitem{#1}{subsection}{\thesubsection}}

\newtheorem{theorem1*}{Theorem}

\newtheoremstyle{named}{}{}{}{}{\bfseries}{}{.5em}{Теорема \thmnote{#3}}
\theoremstyle{named}
\newtheorem*{namedtheorem}{Theorem}

\newcommand{\italicbold}[1]{\emph{\textbf{#1}}}

\newlist{colloq}{enumerate}{1}
\setlist[colloq]{label=\textbf{\arabic*.}}

\everymath{\displaystyle}

\renewcommand*{\arraystretch}{1.5}

\title{\HugeАлгебра, Экзамен 2}
\author{
Анненков Владислав \\
\href{https://teleg.run/Vakosta}{@Vakosta},
\href{https://github.com/Vakosta/hse-tex}{GitHub} \\
}

\usepackage[ddmmyyy,hhmmss]{datetime}
\settimeformat{xxivtime}
\renewcommand{\dateseparator}{.}
\date{Дата изменения: \today \ в \currenttime}

\begin{document}
    \maketitle
    \tableofcontents
    \newpage

    \section{Общая алгебра}
    \begin{colloq}

        \subsection{Важные нюансы}

        \(A \setminus B\) — множество элементов, которые входят в \(A\), но не входят в \(B\).


        \newline

        \subsection{Виды групп}\label{subsec:виды-групп}

        \begin{definition}
            \(Z_n\) — классы целых чисел, имеющих одинаковый остаток при делении на \(n\).
        \end{definition}

        \begin{definition}
            \(D\) — группа Диэдра.
            Группа симметрий правильного n-угольника.\newline
            Обозначение: \(D_n\). \(|D_n| = 2n\), т.к. там есть \(n\) вращений и \(n\) отражений.\newline
            \(D_n = <r, s>\), где \(r^n = id, s^2 = id; s^{-1}rs = r^{-1}\).
        \end{definition}

        \begin{definition}
            \(Q\) — группа кватернионов.
            \(Q_8 = \{ \pm 1, \pm i, \pm j, \pm k \mid i^2 = j^2 = k^2 = ijk = -1 \}\)
        \end{definition}

        \begin{definition}
            \(A\) — знакопеременная группа, симметрическая.
            Т.е. множество всех чётных подстановок.
            \(|A_n| = \frac{n!}{2}\); \(A_n \subset S_n\).

            \(\begin{array}{rrrrrrrrr}
                  * & 1 & -1 & i & -i & j & -j & k & -k \\ \cline{2-9}
                  1 & \multicolumn{1}{|c}{1}   & -1 & i & -i & j & -j & k & -k \\
                  -1 & \multicolumn{1}{|c}{-1} & 1 & -i & i & -j & j & -k & k \\
                  i & \multicolumn{1}{|c}{i}   & -i & -1 & 1 & k & -k & -j & j \\
                  -i & \multicolumn{1}{|c}{-i} & i & 1 & -1 & -k & k & j & -j \\
                  j & \multicolumn{1}{|c}{j}   & -j & -k & k & -1 & 1 & i & -i \\
                  -j & \multicolumn{1}{|c}{-j} & j & k & -k & 1 & -1 & -i & i \\
                  k & \multicolumn{1}{|c}{k}   & -k & j & -j & -i & i & -1 & 1 \\
                  -k & \multicolumn{1}{|c}{-k} & k & -j & j & i & -i & 1 & -1
            \end{array}\)
        \end{definition}

        \newline

        \subsection{Гомоморфизм}

        Отображение \(f:G \to G'\) группы \((G, *)\) в группу \((G,\circ)\) называется гомоморфизмом,
        если \(\forall a, b \in G\) \(\f(a * b) = f(a) \circ f(b)\).

        \underline{Пример:}

        \(\det : GL_n(R) \to R'\) — это \(R \ {0}\) с операцией умножения.
        Это гомоморфизм, т.к. \(\det (A * B) = \det A * \det B\)

    \end{colloq}

\end{document}